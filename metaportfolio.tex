\documentclass{article}
\usepackage{graphicx}
\usepackage[letterpaper, portrait, margin=0.5in]{geometry}
\DeclareGraphicsRule{*}{mps}{*}{}
\begin{document}

\section{(TODO)}
\begin{itemize}
\item consider using ConTeXt instead of \LaTeX.
\end{itemize}
\section{Recursion up the Yin/Yang}
Behold!
\begin{center}
\includegraphics{yin-yang-0}
\end{center}

An explanation of how to get this figure.
Label the circle with reference points: u, d, l, r, c.
To get the figure we'll need several steps, but what
a single step does is to take a path from $d$ to $u$
and change it to another path from $d$ to $u$.
The invariants of the transformation are
\begin{itemize}
\item the path starts at $d$
\item the path ends at $d$
\item the path is contained in the circle
\end{itemize}
Here's the single step, where the red arrow represents the input path:
\begin{center}
\begin{tabular}{cc}
input & output \\
\includegraphics{yin-yang-11} &
\includegraphics{yin-yang-12} \\
\end{tabular}
\end{center}
The new path connects 4 parts
\begin{itemize}
\item an arc `in' from $d$ to $l$
\item 2 smaller rotated copies of the input
\item an arc `out' from $r$ to $u$
\end{itemize}
So, one step takes one path to a more `wiggly' path.

Several steps of the construction look like this, where
the final image is in the background, in green:
\begin{center}
\begin{tabular}{cccc}
\includegraphics{yin-yang-1} &
\includegraphics{yin-yang-2} &
\includegraphics{yin-yang-3} &
\includegraphics{yin-yang-4} \\
\end{tabular}
\end{center}

The last step replaces the red arrow with the (traditional) yin/yang curve.

Before drawing a filled curve we need to close it.
To do so, add a half circle from $u$ through $r$ ending at $d$.

\section{Gray Codes}
sketch
\begin{itemize}
\item show basic snake, with control points
\item show how wiggle stitches together 2 snakes \& some arcs
\item mention the invariant
\end{itemize}

\section{Pixel Graphics}
\subsection{The Mandelbrot Set}
\subsection{Bifurcation}
\section{Sorting}
\subsection{Merge Sort}
\subsection{Quick Sort}
\subsection{Insertion Sort}
\subsection{Selection Sort}
\subsection{Heap Sort}
\section{Trees}
\subsection{Binary Search Trees}
\subsection{Heap from Heapsort}
\section{Tilings}
\subsection{Penrose Kites \& Darts}
\section{Knots}
\section{Proofs Without Words}
\subsection{Homotopy is a Group}
\section{Illusions}
\subsection{Impossible Tribar}
\subsection{Keptelen Font}
\section{Libraries}
\subsection{Complex Numbers}
% Railroad diagrams
% Fritz & Brian's diagrams for higher-order functions
% NP-completeness translations
\end{document}
