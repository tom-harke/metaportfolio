\setuphead[section][style=\bfa]
\setupbodyfontenvironment[default][em=italic]

\starttext
\startsection[title={Penrose Tiling}]

%(
\startuseMPgraphic{star}
for i:=0 upto 4:
draw (1cm,0) rotated (360*i/5) -- (0,0);
endfor;
\stopuseMPgraphic

\placefigure[right]{}{\useMPgraphic{star}}
%)
Consider tiling the plane with tiles having some sort of 5-fold symmetry.
That is, if we focus on some corners we'd see something like ...
The full sweep around 360\textdegree, so each of the 5 corners is $360\textdegree/5 = 72\textdegree$.
You can't make triangles all of whose corners are multiples of $72\textdegree$ since the 3 corners need to sum to $180\textdegree$.
However, if you halve the measure to $36\textdegree$ you could, for example, have a triangle with corners
   $36\textdegree+72\textdegree+72\textdegree$
   or
   $36\textdegree+36\textdegree+108\textdegree$
   .
In fact, these are the only 2 possibilities.

\startuseMPgraphic{name}
pair ooh[], oww[];
ooh1 = (0,0);
ooh2 = (0,1cm);
ooh3 = ooh2 rotatedaround(ooh1,108);
draw ooh1--ooh2--ooh3--cycle;

oww1 = (1cm,0);
oww2-oww1 = (0,1cm);
oww3 = oww2 rotatedaround(oww1,36);
draw oww1--oww2--oww3--cycle;
\stopuseMPgraphic

\placefigure[left]{}{\useMPgraphic{name}}
Let '\cdot' represent $36\textdegree$, '\cdot\cdot' represent $72\textdegree =2\times36\textdegree$, etc.
Then $180\textdegree$ is 5 \cdot's, and the law that a triangle's corners sum to $180\textdegree$ is equivalent to saying that, if we mark the corners with dots, there are exactly 5 dots among the 3 corners.

We're half way to having the Penrose Kites \& Darts.
The 113 triangle is half of a Dart, and the 122 is half of a Kite, but we need to figure out how they fit together.

%(
\startuseMPgraphic{square}
for i:=0 upto 2:
   draw ((i,0)--(i,1)--(i,2)) xscaled 1cm yscaled 5mm;
   draw ((0,i)--(1,i)--(2,i)) xscaled 1cm yscaled 5mm;
endfor;
%for i:=0.5 upto 1.5;
%   draw ((i,0)--(i,1)--(i,2)) xscaled 1cm yscaled 5mm dashed evenly;
%   draw ((0,i)--(1,i)--(2,i)) xscaled 1cm yscaled 5mm dashed evenly;
%endfor;
\stopuseMPgraphic

\startuseMPgraphic{yuck}
for i:=0,0.7,2:
   draw ((i,0)--(i,2)) shifted (0,-3) xscaled 1cm yscaled 5mm;
endfor;

for i:=0,1.5,2:
   draw ((0,i)--(0.7,i)) shifted (0,-3) xscaled 1cm yscaled 5mm;
endfor;
for i:=0,0.7,2:
   draw ((0.7,i)--(2,i)) shifted (0,-3) xscaled 1cm yscaled 5mm;
endfor;
\stopuseMPgraphic

\placefigure[right]{}{\useMPgraphic{square}}
Back up and consider an easier problem: tiling with rectangles.
We could start with a space we want to tile and subdivide it into similar parts.
If the parts are not small enough, repeat.
We'd like to try something similar with 113 and 122 triangles, but first look at some things we want to avoid while dividing.
\placefigure[right]{}{\useMPgraphic{yuck}}
We'd like to avoid subdivisions like in figure~4\footnote{todo}, which has 2 problems:
\startitemize[n,nowhite]
\item the smaller tiles have several different shapes
\item some corners are adjacent to edges
\stopitemize
Alternatively, we can phrase 2 principles:
\startitemize[n,nowhite]
\item have as few tile shapes as possible
\item don't allow T-shaped perimeters, where a corner of one tile is in the middle of an edge of another.
\stopitemize
\stopsection
\stoptext
